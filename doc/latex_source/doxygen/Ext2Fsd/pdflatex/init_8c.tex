\hypertarget{init_8c}{
\subsection{init.c File Reference}
\label{init_8c}\index{init.c@{init.c}}
}


\subsubsection{Detailed Description}
Entry and Unload functions.

The Entry and Unload functions are needed to initialize and unload the driver.

{\tt \#include \char`\"{}ntifs.h\char`\"{}}\par
{\tt \#include \char`\"{}fsd.h\char`\"{}}\par
\subsubsection*{Functions}
\begin{CompactItemize}
\item 
NTSTATUS \hyperlink{init_8c_a1}{Driver\-Entry} (IN PDRIVER\_\-OBJECT Driver\-Object, IN PUNICODE\_\-STRING Registry\-Path)
\begin{CompactList}\small\item\em Entry point of the driver.\item\end{CompactList}\end{CompactItemize}


\subsubsection{Function Documentation}
\index{DriverEntry@{DriverEntry}!init.c@{init.c}}\index{init.c@{init.c}!DriverEntry@{DriverEntry}}\hypertarget{init_8c_a1}{
\index{init.c@{init.c}!DriverEntry@{DriverEntry}}
\index{DriverEntry@{DriverEntry}!init.c@{init.c}}
\paragraph[DriverEntry]{\setlength{\rightskip}{0pt plus 5cm}NTSTATUS Driver\-Entry (IN PDRIVER\_\-OBJECT {\em Driver\-Object}, IN PUNICODE\_\-STRING {\em Registry\-Path})}\hfill}
\label{init_8c_a1}


Entry point of the driver.

This function initializes the driver after loading.

\begin{Desc}
\item[Parameters:]
\begin{description}
\item[{\em Driver\-Object}]DRIVER\_\-OBJECT structure of the lower driver. \item[{\em Registry\-Path}]Unicode string to the driver's registry key. \end{description}
\end{Desc}
\begin{Desc}
\item[Returns:]If the driver initialization was successful. \end{Desc}
