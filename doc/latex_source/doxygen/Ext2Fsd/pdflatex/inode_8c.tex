\hypertarget{inode_8c}{
\subsection{inode.c File Reference}
\label{inode_8c}\index{inode.c@{inode.c}}
}


\subsubsection{Detailed Description}
Functions for inode operations.



{\tt \#include \char`\"{}ext2.h\char`\"{}}\par
{\tt \#include \char`\"{}io.h\char`\"{}}\par
{\tt \#include \char`\"{}stdio.h\char`\"{}}\par
{\tt \#include \char`\"{}stdlib.h\char`\"{}}\par
{\tt \#include \char`\"{}string.h\char`\"{}}\par
{\tt \#include \char`\"{}time.h\char`\"{}}\par
\subsubsection*{Functions}
\begin{CompactItemize}
\item 
NTSTATUS \hyperlink{inode_8c_a14}{Ext2Fsd\-Create\-Inode} (\hyperlink{struct__FSD__IRP__CONTEXT}{PFSD\_\-IRP\_\-CONTEXT} Irp\-Context, PFSD\_\-VCB Vcb, PFSD\_\-FCB p\-Parent\-Fcb, ULONG Parent\-Inode\-Number, ULONG Type, ULONG File\-Attr, PUNICODE\_\-STRING File\-Name)
\begin{CompactList}\small\item\em Creates new Inode on disk and makes Directory entry.\item\end{CompactList}\item 
BOOLEAN \hyperlink{inode_8c_a15}{Ext2Fsd\-New\-Inode} (\hyperlink{struct__FSD__IRP__CONTEXT}{PFSD\_\-IRP\_\-CONTEXT} Irp\-Context, PFSD\_\-VCB Vcb, ULONG Group\-Hint, ULONG Type, PULONG Inode\-Number)
\begin{CompactList}\small\item\em Allocate a new Inode on the partition.\item\end{CompactList}\item 
NTSTATUS \hyperlink{inode_8c_a16}{Ext2Fsd\-Add\-Entry} (IN \hyperlink{struct__FSD__IRP__CONTEXT}{PFSD\_\-IRP\_\-CONTEXT} Irp\-Context, IN PFSD\_\-VCB Vcb, IN PFSD\_\-FCB Dcb, IN ULONG File\-Type, IN ULONG Inode\-Number, IN PUNICODE\_\-STRING File\-Name)
\begin{CompactList}\small\item\em Add a new directory entry.\item\end{CompactList}\item 
BOOLEAN \hyperlink{inode_8c_a17}{Ext2Fsd\-Delete\-Inode} (\hyperlink{struct__FSD__IRP__CONTEXT}{PFSD\_\-IRP\_\-CONTEXT} Irp\-Context, PFSD\_\-VCB Vcb, ULONG Inode, ULONG Type)
\begin{CompactList}\small\item\em Free an inode on the disk.\item\end{CompactList}\item 
NTSTATUS \hyperlink{inode_8c_a18}{Ext2Fsd\-Remove\-Entry} (IN \hyperlink{struct__FSD__IRP__CONTEXT}{PFSD\_\-IRP\_\-CONTEXT} Irp\-Context, IN PFSD\_\-VCB Vcb, IN PFSD\_\-FCB Dcb, IN ULONG Inode)
\begin{CompactList}\small\item\em Remove a directory entry.\item\end{CompactList}\item 
BOOLEAN \hyperlink{inode_8c_a19}{Ext2Fsd\-Save\-Group} (IN \hyperlink{struct__FSD__IRP__CONTEXT}{PFSD\_\-IRP\_\-CONTEXT} Irp\-Context, IN PFSD\_\-VCB Vcb)
\begin{CompactList}\small\item\em Write group descriptor to the disk.\item\end{CompactList}\item 
BOOLEAN \hyperlink{inode_8c_a20}{Ext2Fsd\-Save\-Inode} (IN \hyperlink{struct__FSD__IRP__CONTEXT}{PFSD\_\-IRP\_\-CONTEXT} Irp\-Context, IN PFSD\_\-VCB Vcb, IN ULONG Inode, IN inode\_\-t $\ast$inode)
\begin{CompactList}\small\item\em Write inode to the disk.\item\end{CompactList}\item 
BOOLEAN \hyperlink{inode_8c_a21}{Ext2Fsd\-Save\-Super} (IN \hyperlink{struct__FSD__IRP__CONTEXT}{PFSD\_\-IRP\_\-CONTEXT} Irp\-Context, IN PFSD\_\-VCB Vcb)
\begin{CompactList}\small\item\em Write super block to the disk.\item\end{CompactList}\item 
BOOLEAN \hyperlink{inode_8c_a22}{Ext2Fsd\-Get\-Inode\-Lba} (IN PFSD\_\-VCB vcb, IN ULONG inode, OUT PLONGLONG offset)
\begin{CompactList}\small\item\em Get offset on physical partition.\item\end{CompactList}\end{CompactItemize}


\subsubsection{Function Documentation}
\index{Ext2FsdAddEntry@{Ext2FsdAddEntry}!inode.c@{inode.c}}\index{inode.c@{inode.c}!Ext2FsdAddEntry@{Ext2FsdAddEntry}}\hypertarget{inode_8c_a16}{
\index{inode.c@{inode.c}!Ext2FsdAddEntry@{Ext2FsdAddEntry}}
\index{Ext2FsdAddEntry@{Ext2FsdAddEntry}!inode.c@{inode.c}}
\paragraph[Ext2FsdAddEntry]{\setlength{\rightskip}{0pt plus 5cm}NTSTATUS Ext2Fsd\-Add\-Entry (IN \hyperlink{struct__FSD__IRP__CONTEXT}{PFSD\_\-IRP\_\-CONTEXT} {\em Irp\-Context}, IN PFSD\_\-VCB {\em Vcb}, IN PFSD\_\-FCB {\em Dcb}, IN ULONG {\em File\-Type}, IN ULONG {\em Inode\-Number}, IN PUNICODE\_\-STRING {\em File\-Name})}\hfill}
\label{inode_8c_a16}


Add a new directory entry.

\begin{Desc}
\item[Parameters:]
\begin{description}
\item[{\em Irp\-Context}]Pointer to the context of the IRP \item[{\em Vcb}]Volume control block \item[{\em Dcb}]Directory control block \item[{\em File\-Type}]Type of entry (file or directory) \item[{\em Inode\-Number}]Number of inode \item[{\em File\-Name}]String with name of the new entry\end{description}
\end{Desc}
\begin{Desc}
\item[Returns:]If the routine succeeds, it must return STATUS\_\-SUCCESS. Otherwise, it must return one of the error status values defined in ntstatus.h. \end{Desc}
\index{Ext2FsdCreateInode@{Ext2FsdCreateInode}!inode.c@{inode.c}}\index{inode.c@{inode.c}!Ext2FsdCreateInode@{Ext2FsdCreateInode}}\hypertarget{inode_8c_a14}{
\index{inode.c@{inode.c}!Ext2FsdCreateInode@{Ext2FsdCreateInode}}
\index{Ext2FsdCreateInode@{Ext2FsdCreateInode}!inode.c@{inode.c}}
\paragraph[Ext2FsdCreateInode]{\setlength{\rightskip}{0pt plus 5cm}NTSTATUS Ext2Fsd\-Create\-Inode (\hyperlink{struct__FSD__IRP__CONTEXT}{PFSD\_\-IRP\_\-CONTEXT} {\em Irp\-Context}, PFSD\_\-VCB {\em Vcb}, PFSD\_\-FCB {\em p\-Parent\-Fcb}, ULONG {\em Parent\-Inode\-Number}, ULONG {\em Type}, ULONG {\em File\-Attr}, PUNICODE\_\-STRING {\em File\-Name})}\hfill}
\label{inode_8c_a14}


Creates new Inode on disk and makes Directory entry.

\begin{Desc}
\item[Parameters:]
\begin{description}
\item[{\em Irp\-Context}]Pointer to the context of the IRP \item[{\em Vcb}]Volume control block \item[{\em p\-Parent\-Fcb}]FCB of the parent directory \item[{\em Parent\-Inode\-Number}]Inode number of the parent directory \item[{\em Type}]Type of entry (file or directory) \item[{\em File\-Name}]\end{description}
\end{Desc}
\begin{Desc}
\item[Returns:]If the routine succeeds, it must return STATUS\_\-SUCCESS. Otherwise, it must return one of the error status values defined in ntstatus.h. \end{Desc}
\index{Ext2FsdDeleteInode@{Ext2FsdDeleteInode}!inode.c@{inode.c}}\index{inode.c@{inode.c}!Ext2FsdDeleteInode@{Ext2FsdDeleteInode}}\hypertarget{inode_8c_a17}{
\index{inode.c@{inode.c}!Ext2FsdDeleteInode@{Ext2FsdDeleteInode}}
\index{Ext2FsdDeleteInode@{Ext2FsdDeleteInode}!inode.c@{inode.c}}
\paragraph[Ext2FsdDeleteInode]{\setlength{\rightskip}{0pt plus 5cm}BOOLEAN Ext2Fsd\-Delete\-Inode (\hyperlink{struct__FSD__IRP__CONTEXT}{PFSD\_\-IRP\_\-CONTEXT} {\em Irp\-Context}, PFSD\_\-VCB {\em Vcb}, ULONG {\em Inode}, ULONG {\em Type})}\hfill}
\label{inode_8c_a17}


Free an inode on the disk.

\begin{Desc}
\item[Parameters:]
\begin{description}
\item[{\em Irp\-Context}]Pointer to the context of the IRP \item[{\em Vcb}]Volume control block \item[{\em Inode}]Number of inode \item[{\em Type}]Type of entry (file or directory)\end{description}
\end{Desc}
\begin{Desc}
\item[Returns:]If the operation finished successfully. \end{Desc}
\index{Ext2FsdGetInodeLba@{Ext2FsdGetInodeLba}!inode.c@{inode.c}}\index{inode.c@{inode.c}!Ext2FsdGetInodeLba@{Ext2FsdGetInodeLba}}\hypertarget{inode_8c_a22}{
\index{inode.c@{inode.c}!Ext2FsdGetInodeLba@{Ext2FsdGetInodeLba}}
\index{Ext2FsdGetInodeLba@{Ext2FsdGetInodeLba}!inode.c@{inode.c}}
\paragraph[Ext2FsdGetInodeLba]{\setlength{\rightskip}{0pt plus 5cm}BOOLEAN Ext2Fsd\-Get\-Inode\-Lba (IN PFSD\_\-VCB {\em vcb}, IN ULONG {\em inode}, OUT PLONGLONG {\em offset})}\hfill}
\label{inode_8c_a22}


Get offset on physical partition.

\begin{Desc}
\item[Parameters:]
\begin{description}
\item[{\em vcb}]Volume control block \item[{\em inode}]Number of inode \item[{\em offset}]Offset on physical partition\end{description}
\end{Desc}
\begin{Desc}
\item[Returns:]If the operation finished successfully. \end{Desc}
\index{Ext2FsdNewInode@{Ext2FsdNewInode}!inode.c@{inode.c}}\index{inode.c@{inode.c}!Ext2FsdNewInode@{Ext2FsdNewInode}}\hypertarget{inode_8c_a15}{
\index{inode.c@{inode.c}!Ext2FsdNewInode@{Ext2FsdNewInode}}
\index{Ext2FsdNewInode@{Ext2FsdNewInode}!inode.c@{inode.c}}
\paragraph[Ext2FsdNewInode]{\setlength{\rightskip}{0pt plus 5cm}BOOLEAN Ext2Fsd\-New\-Inode (\hyperlink{struct__FSD__IRP__CONTEXT}{PFSD\_\-IRP\_\-CONTEXT} {\em Irp\-Context}, PFSD\_\-VCB {\em Vcb}, ULONG {\em Group\-Hint}, ULONG {\em Type}, PULONG {\em Inode\-Number})}\hfill}
\label{inode_8c_a15}


Allocate a new Inode on the partition.

The algoritm of this functions tries first to allocate an inode in the same block group. If there are no more inodes available it searches inodes in other block groups.

\begin{Desc}
\item[Parameters:]
\begin{description}
\item[{\em Irp\-Context}]Pointer to the context of the IRP \item[{\em Vcb}]Volume control block \item[{\em Group\-Hint}]Number of block group where search starts \item[{\em Type}]Type of entry (file or directory) \item[{\em Inode\-Number}]Number of inode\end{description}
\end{Desc}
\begin{Desc}
\item[Returns:]If the operation finished successfully. \end{Desc}
\index{Ext2FsdRemoveEntry@{Ext2FsdRemoveEntry}!inode.c@{inode.c}}\index{inode.c@{inode.c}!Ext2FsdRemoveEntry@{Ext2FsdRemoveEntry}}\hypertarget{inode_8c_a18}{
\index{inode.c@{inode.c}!Ext2FsdRemoveEntry@{Ext2FsdRemoveEntry}}
\index{Ext2FsdRemoveEntry@{Ext2FsdRemoveEntry}!inode.c@{inode.c}}
\paragraph[Ext2FsdRemoveEntry]{\setlength{\rightskip}{0pt plus 5cm}NTSTATUS Ext2Fsd\-Remove\-Entry (IN \hyperlink{struct__FSD__IRP__CONTEXT}{PFSD\_\-IRP\_\-CONTEXT} {\em Irp\-Context}, IN PFSD\_\-VCB {\em Vcb}, IN PFSD\_\-FCB {\em Dcb}, IN ULONG {\em Inode})}\hfill}
\label{inode_8c_a18}


Remove a directory entry.

\begin{Desc}
\item[Parameters:]
\begin{description}
\item[{\em Irp\-Context}]Pointer to the context of the IRP \item[{\em Vcb}]Volume control block \item[{\em Dcb}]Directory control block \item[{\em Inode}]Number of inode\end{description}
\end{Desc}
\begin{Desc}
\item[Returns:]If the routine succeeds, it must return STATUS\_\-SUCCESS. Otherwise, it must return one of the error status values defined in ntstatus.h. \end{Desc}
\index{Ext2FsdSaveGroup@{Ext2FsdSaveGroup}!inode.c@{inode.c}}\index{inode.c@{inode.c}!Ext2FsdSaveGroup@{Ext2FsdSaveGroup}}\hypertarget{inode_8c_a19}{
\index{inode.c@{inode.c}!Ext2FsdSaveGroup@{Ext2FsdSaveGroup}}
\index{Ext2FsdSaveGroup@{Ext2FsdSaveGroup}!inode.c@{inode.c}}
\paragraph[Ext2FsdSaveGroup]{\setlength{\rightskip}{0pt plus 5cm}BOOLEAN Ext2Fsd\-Save\-Group (IN \hyperlink{struct__FSD__IRP__CONTEXT}{PFSD\_\-IRP\_\-CONTEXT} {\em Irp\-Context}, IN PFSD\_\-VCB {\em Vcb})}\hfill}
\label{inode_8c_a19}


Write group descriptor to the disk.

\begin{Desc}
\item[Parameters:]
\begin{description}
\item[{\em Irp\-Context}]Pointer to the context of the IRP \item[{\em Vcb}]Volume control block\end{description}
\end{Desc}
\begin{Desc}
\item[Returns:]If the operation finished successfully. \end{Desc}
\index{Ext2FsdSaveInode@{Ext2FsdSaveInode}!inode.c@{inode.c}}\index{inode.c@{inode.c}!Ext2FsdSaveInode@{Ext2FsdSaveInode}}\hypertarget{inode_8c_a20}{
\index{inode.c@{inode.c}!Ext2FsdSaveInode@{Ext2FsdSaveInode}}
\index{Ext2FsdSaveInode@{Ext2FsdSaveInode}!inode.c@{inode.c}}
\paragraph[Ext2FsdSaveInode]{\setlength{\rightskip}{0pt plus 5cm}BOOLEAN Ext2Fsd\-Save\-Inode (IN \hyperlink{struct__FSD__IRP__CONTEXT}{PFSD\_\-IRP\_\-CONTEXT} {\em Irp\-Context}, IN PFSD\_\-VCB {\em Vcb}, IN ULONG {\em Inode}, IN inode\_\-t $\ast$ {\em inode})}\hfill}
\label{inode_8c_a20}


Write inode to the disk.

\begin{Desc}
\item[Parameters:]
\begin{description}
\item[{\em Irp\-Context}]Pointer to the context of the IRP \item[{\em Vcb}]Volume control block \item[{\em Inode}]Number of inode \item[{\em inode\_\-t}]Pointer to the inode struct which has to be saved\end{description}
\end{Desc}
\begin{Desc}
\item[Returns:]If the operation finished successfully. \end{Desc}
\index{Ext2FsdSaveSuper@{Ext2FsdSaveSuper}!inode.c@{inode.c}}\index{inode.c@{inode.c}!Ext2FsdSaveSuper@{Ext2FsdSaveSuper}}\hypertarget{inode_8c_a21}{
\index{inode.c@{inode.c}!Ext2FsdSaveSuper@{Ext2FsdSaveSuper}}
\index{Ext2FsdSaveSuper@{Ext2FsdSaveSuper}!inode.c@{inode.c}}
\paragraph[Ext2FsdSaveSuper]{\setlength{\rightskip}{0pt plus 5cm}BOOLEAN Ext2Fsd\-Save\-Super (IN \hyperlink{struct__FSD__IRP__CONTEXT}{PFSD\_\-IRP\_\-CONTEXT} {\em Irp\-Context}, IN PFSD\_\-VCB {\em Vcb})}\hfill}
\label{inode_8c_a21}


Write super block to the disk.

\begin{Desc}
\item[Parameters:]
\begin{description}
\item[{\em Irp\-Context}]Pointer to the context of the IRP \item[{\em Vcb}]Volume control block\end{description}
\end{Desc}
\begin{Desc}
\item[Returns:]If the operation finished successfully. \end{Desc}
