\chapter{Challenges and Problems}
\label{cha:problems}
When we first had the idea of writing a file system driver for Windows, we could not determine the complexity of this task. We were completely inexperienced on this kind of software development.  Furthermore, we were encouraged by the continuing studies diploma thesis where they wrote a read-only file system driver. We thought, it should easily be possible to finish a write function for ext2 and because this seemed not enough for a semester thesis, our objective was to include journalling support as well.

After we had finished collecting all information available and started reading, we realized that the problem might be more complex than we thought. First, we tried to understand the ext2 file system using a hexadecimal printout of a small partition and the book ``Understanding the Linux Kernel'' \cite{understandingKernel}. It took us about $1\frac{1}{2}$ weeks until we understood every bit on the partition. 

After this, we tried to understand the basic concepts of a Windows driver. This turned out to be the hard part. Driver programming for Windows is very different from programming we have known and done before. To get some help on the design, we took the read-only driver mentioned before because this design had to be adopted and obeyed by our write extension. Soon we realized that these people from the continuing studies had the same problem we had. The problem is too complex to understand in such a short time. Even though, they could finish their read-only driver because they had a sample driver from which they could copy the code. Unfortunately this was impossible for our create and write dispatch routine. Although there was a driver from Matt Wu (\cite{ext2fsd}), we could not use his code because the basic design was completely different. Because we had to use the code from the read-only driver, we first had to understand the design and knew what already implemented functions were there for. This took us a long time and we never really finished this task because it was simply too much but we understood the parts which were fundamental for us.

Step for step, we tried to implement our functions (create and write). Because we could not copy anything from a sample driver, we had to write everything ourselves which is very time consuming, though very interesting. 

All problems above lead to the result we achieved. The driver is not completely finished. Although this is kind of disappointing, we learned a lot. It is nearly impossible to complete a file system driver in this time without any sample driver which you can use as template except the programmer has wide experience in driver programming. 

To conclude, we do not consider this work as unsuccessful as it might look like but rather we are glad we could make such a experience. It has enabled us to look more closely on how Windows works internally which can help us to conceive problems faster.