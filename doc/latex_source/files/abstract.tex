\begin{abstract}
\setcounter{page}{3}
\addcontentsline{toc}{chapter}{Abstract}
Ext2 and Ext3 are the most popular file systems on Linux operating systems. Because of the popularity of Linux, more and more people use this operating system besides Windows. The main problem is that data stored on a Linux partition cannot be accessed by Windows and vice versa. WinLFS is a solution for Windows, namely an additional Windows file system driver which is capable of reading from and writing to a Linux partition, which is formatted with the Ext2 or Ext3 file system.

WinLFS is a Windows file system driver based on a driver previously developed in a continuing studies diploma thesis. It is implemented following the IFS (Installable File System) Guide provided by Microsoft. Additionally a file system recognizer was developed. This recognizer finds any known (known are ext2 and ext3) partition on a disk on first access and loads the appropriate driver. Therefore, once installed, any ext2 or ext3 partition can be uses like any file system supported by Microsoft.

Not all functions could be implemented due the complexity and lack of time during this thesis.
\end{abstract}