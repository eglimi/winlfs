\chapter{Personal Reports}
\label{cha:personalReports}

\section{Michael Egli}
Da ich mich allgemein sehr f�r Linux interessiere, fand ich das Projekt eine interessante Herausforderung. Interessant fand ich auch die Tatsache, dass das Projekt Einblick in einen Bereich gibt, mit dem sich verh�ltnism�ssig wenige Personen besch�ftigen. Die Treiberprogrammierung unter Windows hat sich dann aber doch als komplizierter erwiesen, als wir zu Beginn dachten. Hinzu kam, dass es kaum brauchbare Dokumentationen zum Thema Dateisystemtreiber gibt. Trotzdem waren wir sehr motiviert und glaubten, die Ziele die wir uns steckten, erreichen zu k�nnen.

Nach und nach stellte sich heraus, dass es ein riesiger Zeitaufwand ist, bis man �berhaupt die Konzepte der Treiberprogrammierung versteht. Da wir auch keinen Beispieltreiber hatten, an den wir uns halten konnten (den Treiber den es gibt passt nicht zum Design des bestehenden read-only Treibers), mussten wir alles selber programmieren. Das hat aus meiner Sicht den Vorteil, dass wir alles was mir machten verstehen mussten, aus der Sicht der Arbeit hat es aber den Nachteil, dass wir langsam voran kamen. 

Auch wenn wir unsere Ziele teilweise nicht erreicht haben, war das Projekt f�r mich sehr lehrreich und ich glaube auch, dass die Ziele die wir uns gesteckt haben unrealistisch war. Die Ziele w�ren vielleicht dann realistisch, wenn jemand schon \emph{sehr} viel Erfahrung mit Treiberprogrammierung hat, was bei uns nicht der Fall war.

Die Arbeit mit Marc wahr sehr angenehm und ich w�rde sie jederzeit wieder mit ihm machen. Auch als wir x-treme programming praktizierten hat es gut funktioniert. Meiner Meinung nach sogar besser und produktiver, als wenn jeder an seinem St�ck Code arbeitete.

\section{Marc Winiger}
Da ich mich sehr f�r das Opensource Betriebssystem Linux interessiere, nebenbei jedoch meistens trotzdem noch ein Windows installiert habe, war ich sofort sehr begeistert vom Thema dieser Arbeit. Ich sah es als grosse Herausforderung, mich mit der Systemprogrammierung von Windows auseinander zusetzen. Allen Beteiligten, dem Team sowie dem Betreuer, war klar, dass es eine schwierige Aufgabe sein wird, jedoch waren wir sehr motiviert und glaubten daran, dass es m�glich sein wird, das Ziel zu erreichen. Ich finde es sehr schade, dass es uns, trotz dem absehbaren sehr grossen Aufwand, von der Abteilung Informatik leider nicht erlaubt wurde, die Arbeit in einem Dreierteam anzugehen.

Die ersten zwei Wochen nach der Planung des Projekts war es meine Aufgabe einen File System Recognizer zu schreiben, der in der Lage ist, ein Ext2 oder Ext3 Dateisystem auf der Festplatte zu erkennen und diese auch voneinander zu unterscheiden, um dann den ben�tigten Dateisystem Treiber zu laden. Leider stellte sich sehr schnell heraus, dass die Treiberprogrammierung unter Windows ein sehr kompliziertes Thema ist und sehr viel Zeit ben�tigt wird, um sich einzuarbeiten. So konnten wir uns leider nicht auf die Implementation des Ext Dateisystems konzentrieren, was mich etwas mehr interessiert h�tte. Stattdessen hatten wir mit der Windows Treiberprogrammierung zu k�mpfen.

Nachdem ich einen bestehenden Recognizer soweit angepasst hatte, dass er unseren Anforderungen entsprach, widmete ich mich ebenfalls dem eigentlichen Dateisystem Treiber, in welchen sich Michael bereits eingearbeitet hatte. Wir entschieden uns die Arbeit von nun an nach dem Prinzip X-treme Programming fortzuf�hren. Aus meiner Sicht war dies die beste L�sung. Einerseits konnte ich mich so, im doch ziemlich umfangreichen Code, relativ schnell zurechtfinden. Andererseits haben wir auftretende Probleme und Fragen immer sofort diskutiert. Gerade dies war ein grosser Zeitvorteil, weil es sich vorwiegend um Probleme handelte, bei denen man alleine unter Umst�nden Stunden braucht, um sie zu l�sen, der andere aber eventuell gleich eine gute Idee f�r einen L�sungsansatz hat.

Trotz dem teilweisen Misserfolg, war es f�r mich eine sehr interessante und lehrreiche Arbeit, die uns einen sehr tiefen Einblick in Funktionsweisen vom Windows System gab. Es w�re f�r mich eine pers�nliche Genugtuung gewesen, wenn wir es nur geschafft h�tten eine leere Datei oder ein Verzeichnis zu schreiben. Leider hat uns Windows bis zum letzten Moment nur mit Bluescreens ``belohnt''.

Dass Michael mich als seinen Team-Partner ausw�hlte, als wir erfahren hatten, dass die Arbeit nur als Zweierteam durchgef�hrt werden darf, hatte mich sehr gefreut, denn wir haben uns erst durch dieses Projekt n�her kennen gelernt. Ich habe sehr gerne mit ihm zusammen gearbeitet und wir verstehen uns auch privat bestens, so dass wir manchmal nach getaner Arbeit die Rapperswiler ``Wirtschaft(en)'' etwas unterst�tzen konnten.