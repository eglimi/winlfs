\begin{longtable}{|p{4cm} X|}
\caption{\label{tab:useCase02}Use Case UC02}\\
\hline
Use Case Name 					&Mount / Unmount\\
Use Case Number					&UC02\\
Actors							&Windows XP user\\
Purpose							&Mount or unmount a partition.\\
Overview						&If a partition is not mounted automatically at boot time, the actor can manually mount or unmount a ext2 or ext3 partition by using Windows XP file system Manager. Alternatively the actor can mount a partition by using the provided shell program.\\
Formality Type							&casual\\
Priority						&2\\
Cross References				&\textit{Functions:} R1.1, R1.2, R1.3\\ 
Main success scenario		&
\begin{enumerate}
\item System recognizes ext partition on start-up.
\item	System starts file system driver.
\item File system driver mounts ext partition.
\item Actor works with mounted partition.
\end{enumerate}
\\
Alternate scenario				&
\begin{enumerate}
\item	System could not mount ext partition.
\item Actor opens Computer Management Window and clicks on logical device.
\item System shows all partitions (including ext2 and ext3).
\item Actor selects a partition and associate desired drive letter $x$.
\item System mounts partition and shows it in the file browser as drive $x$.
\end{enumerate}
\\
Errors							&\begin{itemize}
								 \item Partition not found
								 \item Type of partition not valid
								 \end{itemize}\\
Preconditions					&Partition is unmounted or mounted.\\
Postconditions					&Partition is mounted or unmounted.\\
\hline
\end{longtable}
