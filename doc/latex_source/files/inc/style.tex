%%Pakete auswaehlen
\usepackage[english]{babel}
%\usepackage[T1]{fontenc}
\usepackage[latin1]{inputenc}
\usepackage{fancyhdr}
\usepackage[pdftex]{graphicx}
\usepackage{color}
\usepackage[pdftex]{hyperref}
\usepackage{url}
\usepackage{here}
\usepackage{rotating}
\usepackage{lscape}
\usepackage{pdfpages}
\usepackage{multirow}
\usepackage[nottoc]{tocbibind}  %literaturverzeichnis und index ins inhaltsverzeichnis
\usepackage{ltxtable}
\usepackage{latexsym}
\usepackage[headinclude]{typearea}  %wenn KOMA - script auskomentieren!
\usepackage[hang,small,bf]{caption2}
%\usepackage{color}
\usepackage{fancyvrb}
\fvset{frame=single,numbers=left,fontsize=\small}

%%bedruckter Bereich festlegen - wenn KOMA - script auskomentieren!
\typearea{9}

%%index erstellen
%\usepackage{makeidx}
\makeindex

\bibliographystyle{is-plain}

\pagestyle{fancy}

\fancypagestyle{plain} {
\fancyfoot{}
\fancyhead{}
\renewcommand{\headrulewidth}{0pt}
}

%%twoside header options
\renewcommand{\chaptermark}[1]{%
\markboth{\chaptername\ \thechapter{}: #1}{}}
\renewcommand{\sectionmark}[1]{%
\markright{\thesection{}: #1}{}}
\fancyfoot{}% Unten nichts
\fancyhead[RE]{\itshape\leftmark}% Rechts auf geraden Seiten=innen
\fancyhead[LO]{\itshape\rightmark}% Links auf ungeraden Seiten=innen
\fancyhead[RO,LE]{\thepage}%Rechts auf ungeraden und links auf geraden Seiten

%%oneside header options - display only chapters
%\fancyfoot{}
%\lhead{\thepage}
%\rhead{\itshape\leftmark}

%% Color settings
\pagecolor{white} %background
\color{black}     % text

%% Attribute f�r das PDF-Dokument
\hypersetup{
	pdftitle = {Thesis Linux Filesystem Driver for Windows, ext2, ext3},
	pdfsubject = {Thesis Linux Filesystems for Windows 2000/XP},
	pdfkeywords = {ext2, ext3, Filesystems, Driver, Linux, Windows },
	pdfauthor = {Michael Egli, michael.egli@hsr.ch, Marc Winiger, marc.winiger@hsr.ch},
	colorlinks=true, 
	linkcolor=blue, 
	citecolor=blue, 
	urlcolor=blue, 
	plainpages=false
}

\newcommand{\highlight}[1]{\emph{#1}}
\newcommand{\command}[1]{\texttt{#1}}

%%additional options

%set depth of toc to 2
\setcounter{tocdepth}{1}

%text on the part page
\makeatletter
\def\@endpart{}
\makeatother

%Sans Serif Schrift
%\renewcommand{\familydefault}{cmss}    
%\renewcommand{\baselinestretch}{1.5}

%Doxygen
\newenvironment{CompactList}
{\begin{list}{}{
  \setlength{\leftmargin}{0.5cm}
  \setlength{\itemsep}{0pt}
  \setlength{\parsep}{0pt}
  \setlength{\topsep}{0pt}
  \renewcommand{\makelabel}{}}}
{\end{list}}
\newenvironment{CompactItemize}
{
  \begin{itemize}
  \setlength{\itemsep}{-3pt}
  \setlength{\parsep}{0pt}
  \setlength{\topsep}{0pt}
  \setlength{\partopsep}{0pt}
}
{\end{itemize}}
\newcommand{\PBS}[1]{\let\temp=\\#1\let\\=\temp}
\newlength{\tmplength}
\newenvironment{TabularC}[1]
{
\setlength{\tmplength}
     {\linewidth/(#1)-\tabcolsep*2-\arrayrulewidth*(#1+1)/(#1)}
      \par\begin{tabular*}{\linewidth}
             {*{#1}{|>{\PBS\raggedright\hspace{0pt}}p{\the\tmplength}}|}
}
{\end{tabular*}\par}
\newcommand{\entrylabel}[1]{
   {\parbox[b]{\labelwidth-4pt}{\makebox[0pt][l]{\textbf{#1}}\\}}}
\newenvironment{Desc}
{\begin{list}{}
  {
    \settowidth{\labelwidth}{40pt}
    \setlength{\leftmargin}{\labelwidth}
    \setlength{\parsep}{0pt}
    \setlength{\itemsep}{-4pt}
    \renewcommand{\makelabel}{\entrylabel}
  }
}
{\end{list}}
\newenvironment{Indent}
  { \begin{list}{} {\setlength{\leftmargin}{0.5cm}}
      \item[]\ignorespaces}
  {\unskip\end{list}}

