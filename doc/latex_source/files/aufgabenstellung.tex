\chapter*{Conceptual Formulation}
\addcontentsline{toc}{chapter}{Aufgabenstellung}

Der Zugriff auf Linux-Partitionen im Ext2 oder Ext3 Format ist von Windows XP standardm�ssig nicht unterst�tzt. So k�nnen dort abgelegte Daten unter Windows weder angeschaut noch ge�ndert werden. Windows XP bietet jedoch die M�glichkeit zus�tzliche Dateisystemtreiber zu laden. Ein beschr�nkt funktionsf�higer Dateisystemtreiber f�r Ext2 wurde in einer Abschlussarbeit des NDS Software Engineering bereits entwickelt. Davon ausgehend soll ein einfach handhabbarer und robuster Dateisystemtreiber realisiert werden, der neben Ext2 auch das protokollierende Dateisystem Ext3 unterst�tzt.

%----------
\section*{Aufgaben}
\label{sec:Tasks}
Es ist ein Dateisystemtreiber f�r Windows XP zu entwickeln, der folgende Eigenschaften aufweist: 
\begin{itemize}
	\item	Unterst�tzung der Dateisystemformate Ext2 und Ext3 
    \item	Lese- und Schreibzugriff
	\item	Sinnvolle Umsetzung der Linux Dateisystemrechte auf Windows Dateisystemrechte 
    \item	Robustes Betrienbsverhalten v.A. hinsichtlich Besch�digung von Ext2/3 Partitionen und Datenverlust
    \item	Manuelles und automatisches Mounting/Unmountiung von Ext2/3 Partitionen
    \item	Optional: gute Zwischenpufferung f�r hohen Durchsatz und kurze Reaktionszeiten
\end{itemize}
Nach Bedarf sind zus�zliche Hilfs- und Testprogramme zu erstellen, die Auskunft geben �ber den Zustand der eingeh�ngten Ext2/3 Partitionen. Erfasste Parameter k�nnten sein: Anzahl geschriebener/gelesener Bytes, Anzahl Verzeichniszugriffe, Anzahl Fehler/Wiederholungen, mittlere Reaktionszeit und Durchsatzrate. Testprogramme sollten  ein automatisiertes Pr�fen von Ext2/3 Partitionen erlauben, um einfach festzustellen ob der Dateisystemtreiber einwandfrei l�uft bzw. Probleme mit den entsprechenden Partitionen festgestellt wurden. F�r die Entwicklung des Treibers stehen der IFS(Inetrface File System)-Kit von Microsoft, sowie alle ben�tigten Entwicklertools zur Verf�gung.

Als Resultat soll eine demonstrierbare und in der Praxis einsetzbare Software vorliegen, deren Entwurf und Funktionsweise in einem Bericht dokumentiert sind. Die SW soll auf verschiedensten PC's funktionsf�hig sein und mit einer Bedienungsanleitung erg�nzt werden.

Es besteht die Absicht den entwickelten Treiber gratis interessierten Personen zug�nglich zu machen, wobei aber die Entwicklung innerhalb der Lizenzvorgaben von Microsoft zu erfolgen hat.


%----------
\section*{Technologien}
\begin{itemize}
	\item	Festplattenspeicherung auf logischer Ebene
    \item	Datensysteme
    \item	Windows XP Treiber, spez. Dateisystemtreiber 
\end{itemize}

\vspace*{50mm}
\noindent
\textbf{Unterschrift} \dotfill 